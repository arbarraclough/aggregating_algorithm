\section{Introduction \textit{(1,000)}}\label{section:Introduction}

\begin{quote}
    The introduction should set the stage for the dissertation. It should provide background information on Online Prediction and the Aggregating Algorithm, highlihgt the importance of the study, and outline the research questions or hypotheses. Additionally, this section should briefly describe the structure of the dissertation.
\end{quote}

\noindent\rule{\textwidth}{0.1pt}

\begin{itemize}
    \item Outline what you are investigating in the project.
    \item Why does this subject area interest you?
    \item How will this project help you in your future career?
    \item Give a summary of the project as a whole, giving a preamble before explaining the nomenclature in the following sections.
    \item Explain the structure of the report, making it easy for examiners and markers to see what you are aiming to achieve with this project.
\end{itemize}

\noindent\rule{\textwidth}{0.1pt}

\subsection{Project Scope and Objectives}


\subsection{Motivation and Interest in the Subject Area \textbf{(259)}}
The motivation for selecting a project in this subject area is rooted in both my personal and professional interests, as well as the discussions I had with my now-supervisor, Dr.\ Yuri Kalnishkan, before finalising my selection.

During this academic year, the module that most piqued my interest was CS5200 \textendash\ On-line Machine Learning because I was interested in the techniques that allowed machine learning models to gradually improve over time as more data became available to them without the need to retrain the model on the entire newly-updated dataset; something that had not been covered previously by other modules. Due to the module's small size and frequent absentees, I was able to gain a deeper understanding of the module, in large part due to Dr.\ Kalnishkan's willingness to explain portions of the syllabus in extreme detail. Alongside the lectures, I felt like I was strongly suited to the contents of the module because it has strong ties to the field of statistics \textendash\ another area that I thoroughly enjoyed throughout my education. 

Regarding my professional aspirations, I am set to begin my career later this year and I am of the firm belief that the work that I have done in this subject area is highly relevant, not only to the job I am to start in September, but also for my career plan due to its relevance across a variety of industries \textendash\ including finance, energy, and insurance.

Ultimately, the combination of all of these factors led me to pursue a project further investigating on-line prediction, and prediction with expert advice.

\subsection{Structure of the Dissertation \textbf{(265)}}
The dissertation is split into distinct chapters, each dedicated to exploring a specific aspect of the work. The following outline guides the reader through the report by providing a brief overview of the contents of each chapter.

Chapters 2 through 5 contain a literature review organised to explain the concepts that the practical portion of the dissertation aims to explore. \hyperref[section:On-line_Prediction]{Chapter 2} defines the problem of On-line Prediction, outlining the scenarios that it applies to, and the protocols that such problems follow. Additionally, it explores how on-line learning differs from traditional batch learning and defines concepts that will be critical to understanding the following sections. \hyperref[section:Prediciton_with_Expert_Advice]{Chapter 3} introduces the problem of Prediction with Expert Advice, explaining its significance and applications in the real world, as well as exploring some algorithms that are used to solve such problems \textendash\ including their theoretical bounds. \hyperref[section:Aggregating_Algorithm]{Chapter 4} introduces the Aggregating Algorithm that this report is centred around, exploring how it differs from other methods of Prediction with Expert Advice. \hyperref[section:Specialist_Experts]{Chapter 5} focuses on Specialist Experts, defining what they are and how the base Aggregating Algorithm must be modified to accommodate them.

\hyperref[section:Practical]{Chapter 6} contains the practical portion of the dissertation, explaining how the research problem was handled based on the concepts explored in the literature review, the findings from conducting the requirements analysis and design processes, and the results found when comparing an individual's idea of ``random'' to that of a random number generator.

Finally, \hyperref[section:Conclusion]{Chapter 7} contains a conclusion which discusses the findings of the investigation as well as a self-evaluation of my performance throughout the project.