\section{Introduction}\label{section:introduction}

\subsection{Project Scope and Objectives}
The aim of this project is to implement the Aggregating Algorithm for Specialist \textit{(Sleeping)} Experts, a method of Prediction with Expert Advice, to scenarios involving human-generated sequences and evaluate its predictive performance in pre-empting what the human subject will input, effectively testing how good human subjects are at generating random inputs.

As an introduction to the concepts that will be explored in the sections to come, this algorithm allows for the effective pooling of different prediction algorithms, known as `Experts', in order to improve the algorithm's prediction accuracy. By aggregating several predictions, this allows the final prediction outputted by the algorithm to be nearly as accurate as the best-performing Expert.

This project will encompass several key areas, including:
\begin{itemize}
    \item \textbf{Explaining the Theory of Perceived Randomness.} The basis of this study revolves around the human perception of randomness, which is different to objective randomness, therefore the underlying psychological mechanisms for how humans judge and perceive randomness must be understood.
    \item \textbf{Explaining the Theory of Prediction with Expert Advice.} The other portion of this study is firmly based in the subject matter of Prediction with Expert Advice, primarily focussing on the Aggregating Algorithm and the Aggregating Algorithm for Speciallist Experts so the underlying theory must be explored with a thorough review of the current literature.
    \item \textbf{Implementing the Aggregating Algorithm.} This project will primarily investigate the Aggregating Algorithm introduced by Vovk (see~\cite{vovk:1990},\ \cite{vovk:1998}).
    \item \textbf{Handling Specialist Experts.} Introduced by Freund~\cite{freund:1997}, \textit{Specialist Experts} may refrain from making predictions at certain points, meaning that the Aggregating Algorithm has to be modified slightly~\cite{kalnishkan:2015}.
    \item \textbf{Evaluating the Performance of Human Subjects in Generating Statistically Random Seuqences.} Through conducting the experiment outlined in this paper, this study will present how well the subjects were able to generate a ``random'' sequence when compared to the statistical definition used by statisticians.
    \item \textbf{Evaluating the Performance of the Algorithm in Predicting Human-Generated Outcomes.} The experiment will also compare how well the Aggregating Algorithm for Specialist Experts was able to pre-empt each subject's sequences, in a somewhat adversarial comparison to statistical randomness.
\end{itemize}

\subsection{Motivation and Interest in the Subject Area}
The motivation for selecting a project in this subject area is rooted in both my personal and professional interests, as well as the discussions I had with my now-supervisor, Dr.\ Yuri Kalnishkan, before finalising my selection.

During this academic year, the module that most piqued my interest was CS5200 \textendash\ On-line Machine Learning because I was interested in the techniques that allowed machine learning models to gradually improve over time as more data became available to them without the need to retrain the model on the entire newly-updated dataset; something that had not been covered previously by other modules. Due to the module's small size and frequent absentees, I was able to gain a deeper understanding of the module, in large part due to Dr.\ Kalnishkan's willingness to explain portions of the syllabus in extreme detail. Alongside the lectures, I felt like I was strongly suited to the contents of the module because it has strong ties to the field of statistics \textendash\ another area that I thoroughly enjoyed throughout my education.

Regarding my professional aspirations, I am set to begin my career later this year and I am of the firm belief that the work that I have done in this subject area is highly relevant, not only to the job I am to start in September, but also for my career plan due to its relevance across a variety of industries \textendash\ including finance, energy, and insurance.

Ultimately, the combination of all of these factors led me to pursue a project investigating on-line prediction, and prediction with expert advice.

\subsection{Structure of the Dissertation}
The following study is split into several distinct chapters and subsections, each dedicated to exploring a specific aspect behind the study being conducted. The following outline guides you, the reader, through the report by providing a brief overview of the contents of each chapter.

Chapter~\ref{section:literature_review} contains the Literature Review that is organised to explain the underlying theory that the practical portion of this study aims to investigate.

Section~\ref{subsection:perceived_randomness} explores the human perception of randomness and how it is, in fact, different from the objective definition of randomness, and aims to provide context to answer the question ``Are humans good randomisers?'' in both judgement and generation scenarios.

Section~\ref{subsection:prediction_with_expert_advice} is diverse in its contents. Subsection~\ref{subsubsection:introduction_to_on-line_prediction} defines the problem of On-line Prediction, outlining the scenarios in which it is applicable, and the protocols that such problems follow. Additionally, it explores how On-line Prediction differs from the traditional Machine Learning frameworks of Batch Learning and Timeseries Analysis and defines concepts that will be critical to understanding the experiment being conducted. Subsection~\ref{subsubsection:prediction_with_expert_advice} expands on the information presented in the previous subsection by defining the modifications to the original framework to now incorporate the predictions made by a pool of Experts and how the Learner aggregates those to inform its own. It also gives two examples of merging strategies that could be used to accomplish that task. Subsection~\ref{subsubsection:aggregating_algorithm} introduces the algorithm that is the main focus of this study, delving into the rationale behind the algorithm and discussing the bounds guaranteed by utilising it. Lastly, Subsection~\ref{subsubsection:aggregating_algorithm_for_specialist_experts} defines a modification to the original Aggregating Algorithm that allows it to be used in scenarios involving Specialist Experts, a term used to define any Expert that has the ability to abstain from making a prediction.

Chapter~\ref{section:experiment_design_and_methodology} is centred on the practical applications of the theory and how it was used to derive the experiment that is to be conducted, including how the experiment was designed, how the Aggregating Algorithm was applied to the proposed scenario and how the data gathered from the subjects will be analysed.

Chapter~\ref{section:analysis_of_perceived_randomness} will discuss the findings of the study in detail, ultimately attempting to answer the hypothesied question ``are humans good randomisers?''.

Finally, Chapter~\ref{section:conclusion} contains a conclusion that summarises the findings of the study, as well as a self-evaluation of the project.
