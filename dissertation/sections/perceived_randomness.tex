\subsection{Perceived Randomness}
\subsubsection{Introduction to Perceived Randomness}
\textbf{Definition:} Explanation of what perceived randomness is.\newline
\textbf{Importance:} Discussion on why perceived randomness is significant in various fields.\newline
\noindent\rule{\textwidth}{0.1pt}
Perceived Randomness refers to the human tendency to asses sequences of events or data as random or non-random based on subjective criteria. This perception is often influenced by cognitive biases and heuristics, which can lead to misjudgements. Humans typically look for patterns or irregularities in sequences and may perceive a truly random sequence as non-random if it doesn't match their expectation of what randomness should look like.

Binary sequences are sequences composed of two distinct symbols, $0$ and $1$. When evaluating the randomness of such sequences, people often expect certain characteristics, such as:
\begin{itemize}
    \item A roughly equal number of $0$s and $1$s.
    \item No long runs of identical symbols.
    \item A lack of obvious patterns or regularities.
\end{itemize}
However, truly random binary sequences can occasionally contain runs of identical symbols or other patterns that might appear non-random to an observer. People's perception of randomness in binary sequences is therefore influenced by these expectations and may not always align with actual randomness.\newline
\noindent\rule{\textwidth}{0.1pt}

\subsubsection{Randomness in Binary Sequences}
\textbf{Human vs. Algorithmic Generation}\newline
\textbf{Human Perception:} How humans perceive randomness.\newline
\textbf{Algorithmic Methods:} Comparison of human and algorithmic sequence generation.\newline
\noindent\rule{\textwidth}{0.1pt}
When dealing with generated binary sequences, such as those produced by random number generators or algorithms, people apply the same subjective criteria to judge randomness. Even though these sequences are designed to be random, they may sometimes exhibit patterns or anomalies that seem non-random. This perception is shaped by the same cognitive biases that affect judgements of naturally occurring sequences. Evaluating the randomness of generated binary sequences often involves statistical tests and analysis to ensure they meet the mathematical criteria for randomness, which can differ from human perceptions.\newline
\noindent\rule{\textwidth}{0.1pt}

\noindent\textbf{Methods for Generating Sequences}\newline
\textbf{Techniques:} Different methods for generating binary sequences.\newline
\textbf{Comparative Analysis:} Evaluation of these methods in terms of perceived randomness.\newline
\noindent\rule{\textwidth}{0.1pt}