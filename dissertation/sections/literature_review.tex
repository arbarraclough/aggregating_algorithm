\section{Literature Review}\label{section:literature_review}
\subsection{Introduction}
Through exploring the human capacity for judging and generating random binary sequences, this Literature Review will examine a variety of studies that delve into the complex, and somewhat subconscious, relationship between what a human perceives as random and what is truly random according to the statistical definition.

The term `random'' inherently means something that is happening by chance with no cause or reason, however, the way in which statisticians measure randomness is through statistical tests, ultimately meaning that there is some measurable quantity to determine whether something conforms to what we objectively consider random. That being said, humans have a subjective view on randomness which may deviate significantly from what is considered objective randomness. The discrepancy between objective and subjective randomness forms the basis of this study, particularly in the context of whether humans are capable of producing sequences that are statistically indistinguishable from those that are expected of a random process.

A fundamental debate that is found throughout the literature reviewed in this study is whether humans, when asked to perceive and generate random sequences, can truly do so without identifying ``patterns'' that happen occur by coincidence regarding judgement tasks, or without introducing subconscious patterns and biases regarding production tasks. Early research conducted by Reichenbach, with subsequent studies performed by Ross, Wagenaar, Nickerson (to name a few), highlight a complex and often contradictory knowledge base with some of their findings indicating that humans are inherently poor at identifying and generating random sequences, while others suggest the opposite in that humans might approximate randomness better than previously thought.

This review does not aim to provide a definitive answer to the question, however, it does synthesise the current understanding of how humans perceive randomness, as well as connects these findings to the broader context of prediction algorithms\textemdash{}by understanding the limitations of human-generated randomness, insights can be gained into how these subconscious biases might cause algorithms specifically designed for prediction in scenarios with incomplete information, such as the Aggregating Algorithm, to perform better than would be expected of a truly random process. This hypothesis forms the basis of the experimental work performed in this study.

\subsection{Perceived Randomness}
\subsubsection{Are Humans Good Randomisers?}
The definition of the term ``random'' is a contentious area for debate. In essence, randomness is an unobservable characteristic of a generating process therefore the act of trying to define it is somewhat contradictory. In order to determine if a process or sequence is random, it has to be put through statistical tests for specific properties which are deemed to be ``random''. However, because these tests are statistical in nature, are the conclusions drawn objectively random or purely subjective?

Research into the human perception and generation of random sequences is a common topic within psychological papers, yet the contradictory nature of findings results in a less-than-satisfactory answer to the question of ``Are humans good randomisers?'' (much like when defining the term itself).

The origins of such a question can be traced back to an observation made by Hans Reichenbach in The Theory of Probability, 1949~\cite{reichenbach:1949}; he suggested that when asked to produce a series that seemed random to them, people untrained in the theory of probability would be unable to generate such a series and, instead, generate one that would contain patterns and biases, e.g.\ too many alternations than what was expected. This ultimately suggests that humans are not good randomisers which is the prevalent opinion to date. This behaviour is attributed to the fact that human-generated sequences often reflect the underlying psychological tendencies of subjects, rather than the unpredictability of true randomness.

While Reichenbach assumes the stance that humans are not good randomisers, the alternative to this conclusion was put forward by the work of Bruce Ross~\cite{ross:1955}. Ross explores the processes involved in randomising binary sequences and analyses the methods that people used, as well as the typical mistakes that they made, when attempting to create random sequences. In his study, Ross got 60 subjects to stamp cards with either an `$\mathcal{O}$' or an `$\mathcal{X}$' and  place them singly in a $100$-item sequence in the middle of a table that they thought to be random, with item frequencies of either $50-50$, $60-40$, or $70-30$. These sequences were then scored against the expected properties of a random sequence and, based on the analysis conducted, resulting in the conclusion that ``the prevalent \textit{a priori} assumption that the human being is a systematically biased randomi[s]er was not borne out.''~\cite{ross:1955}

\subsubsection{Judgement vs. Production of Random Binary Sequences}
As alluded to by the question posed in the previous subsection, the human perception of randomness is a subjective, rather than objective, quality. Because of this subjectivity, two natural interpretations can follow—either that subjects have an incorrect idea of what randomness is and should look like, or that subjects intuitively know what true randomness should look like but some internal functional limitation prevents the judgement and production of such sequences~cite{wagenaar:1970}, being so powerful that subjects may choose to forego statistical analysis in favour of this intuitive ``gut feeling''~cite{bar-hillel:1991}. Since the main topic of this dissertation is Prediction with Expert Advice (to be introduced in the following subsection) for $\eta$-mixable games, the focus of this literature review will be on the judgement and production of random binary sequences. Both types of study make interesting observations about the internal mechanism responsible for the human perception of randomness, namely ``that [humans] see clumps or streaks in truly random series and expect more alternation, or shorter runs, than are there'' and ``[humans] produce series with higher than expected alternation rates''~\cite{bar-hillel:1991}.

We will first explore judgement. In Willem Wagenaar's 1970 study titled ``Appreciation of conditional probabilities in binary sequences'', he examines people's comprehension of conditional probabilities, focusing on how people perceive and interpret the likelihood of certain events occurring given previous outcomes~\cite{wagenaar:1970}. Through conducting this experiment, this research reveals a significant gap in understanding and the systematic biases that affect our judgements of conditional probabilities.

The study controlled the conditional probability of a $0$ after $0$ ($1$ after $1$) as the experimental variable, testing it in the range $0.2-0.8$ with $0.1$ increments, i.e. 7 values of conditional probability, for first-, second-, and third-orders of dependency. Participants were then shown 16 sets of 7 binary sequences (generated with one of these values of conditional probabilities) for each order of dependency and asked to select and record the sequence that looked the most random to them—explained as the sequence that looked the most likely to be produced when flipping a fair coin. For reference, in a truly random binary sequence, the conditional probability of $0$ after $0$ ($\text{Pr}(0|0)$) or $1$ after $1$ ($\text{Pr}(1|1)$) for the first order of dependency is $0.5$. However, Wagenaar identified that conditional probabilities close to $0.4$ were perceived as the most random across all orders of dependencies, affirming the position that humans aren't good randomisers and that there is some subjective concept of randomness. This study also highlights the bias in favour of `negative recency', more commonly known as the gambler's fallacy wherein gamblers will tend to bet on red after a run of blacks (and vice versa) on a roulette wheel, ultimately causing humans to favour series with slightly more alternations than is expected of true randomness. He postulates that this is likely to be because subjects ``cannot process such a mathematical quantity as `conditional probability'\ldots Rather, they will look at some other characteristics like, for instance, the run-structure of the sequence''~\cite{wagenaar:1970}.

% \subsubsection{Introduction to Perceived Randomness}
% \textbf{Definition:} Explanation of what perceived randomness is.\newline
% \textbf{Importance:} Discussion on why perceived randomness is significant in various fields.\newline
% \noindent\rule{\textwidth}{0.1pt}
% Perceived Randomness refers to the human tendency to asses sequences of events or data as random or non-random based on subjective criteria. This perception is often influenced by cognitive biases and heuristics, which can lead to misjudgements. Humans typically look for patterns or irregularities in sequences and may perceive a truly random sequence as non-random if it doesn't match their expectation of what randomness should look like.

% Binary sequences are sequences composed of two distinct symbols, $0$ and $1$. When evaluating the randomness of such sequences, people often expect certain characteristics, such as:
% \begin{itemize}
%     \item A roughly equal number of $0$s and $1$s.
%     \item No long runs of identical symbols.
%     \item A lack of obvious patterns or regularities.
% \end{itemize}
% However, truly random binary sequences can occasionally contain runs of identical symbols or other patterns that might appear non-random to an observer. People's perception of randomness in binary sequences is therefore influenced by these expectations and may not always align with actual randomness.\newline
% \noindent\rule{\textwidth}{0.1pt}

% \subsubsection{Randomness in Binary Sequences}
% \textbf{Human vs. Algorithmic Generation}\newline
% \textbf{Human Perception:} How humans perceive randomness.\newline
% \textbf{Algorithmic Methods:} Comparison of human and algorithmic sequence generation.\newline
% \noindent\rule{\textwidth}{0.1pt}
% When dealing with generated binary sequences, such as those produced by random number generators or algorithms, people apply the same subjective criteria to judge randomness. Even though these sequences are designed to be random, they may sometimes exhibit patterns or anomalies that seem non-random. This perception is shaped by the same cognitive biases that affect judgements of naturally occurring sequences. Evaluating the randomness of generated binary sequences often involves statistical tests and analysis to ensure they meet the mathematical criteria for randomness, which can differ from human perceptions.\newline
% \noindent\rule{\textwidth}{0.1pt}

% \noindent\textbf{Methods for Generating Sequences}\newline
% \textbf{Techniques:} Different methods for generating binary sequences.\newline
% \textbf{Comparative Analysis:} Evaluation of these methods in terms of perceived randomness.\newline
% \noindent\rule{\textwidth}{0.1pt}
\newpage

\subsection{Prediction with Expert Advice}\label{subsection:prediction_with_expert_advice}
While understanding Perceived Randomness highlights the limitations of human intuition in recognising and generating random sequences, these insights naturally extend to the domain of prediction, particularly in scenarios where decisions must be made with some level of uncertainty. Just as individuals struggle to perceive randomness accurately, the issue may extend to how they generate ``random'' sequences, accidentally creating predictable patterns that may emerge as a result of subconscious biases they are unaware of. This is where the concept of Prediction with Expert Advice becomes relevant. The following sections will explore the mechanics of On-line Prediction and Prediction with Expert Advice, then introduce two methods used to combine the advice from a pool of Experts in order to improve prediction accuracy.

\subsubsection{Introduction to On-line Prediction}\label{subsubsection:introduction_to_on-line_prediction}
Within the areas of Machine Learning and Statistics, there lies the problem of accurately ``predicting future events based on past observations''~\cite{cesa-bianchi:1997} known as On-line Prediction\textemdash{}the framework for which is displayed below. This problem refers to methods where a model makes predictions sequentially and updates its parameters in real-time as new data points become available.

There is a particular class of algorithm that is designed to tackle this, with one of the most notable being the ``Strong'' Aggregating Algorithm proposed by Volodymyr Vovk~\cite{vovk:1990} which forms the basis of this study. The adjective ``Strong'' is emphasised with speech marks to help distinguish it from the ``Weak'' Aggregating Algorithm proposed by Yuri Kalnishkan and Michael Vyugin~\cite{kalnishkan/vyugin:2008} that will not be explored in detail in this paper.

Given that the foundations of this study lie firmly within On-line Prediction, this section aims to lay a comprehensive foundation, exploring the key concepts and frameworks that will set the stage for the discussions in Chapter~\ref{section:analysis_of_perceived_randomness}.

\begin{protocol}[H]
    \caption{On-line Prediction Framework}\label{on-line_prediction_framework}
    \begin{algorithmic}[1]
        \State{FOR $t = 1, 2, \ldots$}
        \State{\hspace{\algorithmicindent} learner $L$ outputs $\gamma_t \in \Gamma$}
        \State{\hspace{\algorithmicindent} nature outputs $\omega_t \in \Omega$}
        \State{\hspace{\algorithmicindent} learner $L$ suffers loss $\lambda(\gamma_t, \omega_t)$}
        \State{END FOR}
    \end{algorithmic}
\end{protocol}

\paragraph{On-line Prediction, Batch Learning and Timeseries Analysis}\label{paragraph:on-line_prediction_batch_learning_and_timeseries_analysis}
In order to effectively explain On-line Prediction, we must first compare and contrast it with some alternative frameworks used in Machine Learning: Batch Learning and Timeseries Analysis.

First, we will explore the distinction between the On-line Prediction and Batch Learning frameworks. With Batch Learning, a whole training set of labelled examples $(x_i, y_i)$ is given to the Learner at once and used to train a model. In contrast, On-line Prediction involves gradually feeding the Learner information over time, requiring any models made to continuously adapt to the new data it is given whilst also requiring the Learner to take actions based on the incomplete information it currently possess rather than waiting for a complete picture~\cite{kalnishkan:2015}. This forced adaptability ensures that the predictions outputted by an On-line Prediction algorithm remain accurate to the information that the model deems as relevant as it continues to gain additional information. This fact makes these models particularly valuable in applications that require immediate responses and flexibility in predictions, such as financial market analysis and weather forecasting.

The second distinction we will explore is between On-line Prediction and Timeseries Analysis since, while they are both ways of handling sequential data, they are unique. On-line Prediction is based on processing data points sequentially and updating a predictive model in real-time whereas Timeseries Analysis is based on modelling and forecasting data that is collected over successive time intervals. The prior approach does not impose any strict assumptions about the underlying data-generating process, even going so far as to not assume the existence of such a process~\cite{vovk:2001}, while the latter assumes a structured approach in which the current observation is dependent on the previous observations. This assumption leads to the data-generating processes being modelled with stochastic processes, such as \textit{autoregressive integrated moving average (ARIMA)} or \textit{state-space models}~\cite{box:2015}.

The majority of literature on On-line Prediction takes a similar stance that no assumptions can be made about the sequence of outcomes that are observed and, because of this, the analyses are done over the worst-case scenario and may, in fact, be better in reality~\cite{cesa-bianchi:1997}.

\paragraph{Notation}\label{paragraph:notation}
Having now clearly defined what On-line Prediction is and is not, we can delve into the notation Protocol~\ref{on-line_prediction_framework} presents.

Consider a scenario where the elements of a sequence, known as \textbf{\textit{outcomes}} $\omega_t$, occur at discrete time steps $t \in T$ $\omega_1, \omega_2, \ldots, \omega_T$ which we assume to be drawn from a known \textbf{\textit{outcome space}} $\Omega$. In this scenario, the Learner is tasked with making \textbf{\textit{predictions}} $\gamma_t$ about these outcomes before they occur. Similarly to the outcomes themselves, we assume that the Learner's predictions are drawn from a known \textbf{\textit{prediction space}} $\Gamma$ which may or may not be the same as the outcome space $\Omega$.

Once the Learner has made their prediction, the true outcome is revealed and the quality of the Learner's prediction is measured by \textbf{\textit{loss function}} $\lambda(\gamma_t, \omega_t)$. In essence, this function measures the discrepancy between the prediction and the outcome—known as ``regret'', meaning it quantifies the effect of when the prediction $\gamma_t$ is confronted with the outcome $\omega_t$ in hindsight. It does this by mapping the input space $\Gamma \times \Omega$ to a subset of the real-number like $mathbb{R}$, typically $[0, +\infty)$~\cite{kalnishkan:2009}.

\begin{equation}
    \lambda(\gamma_t, \omega_t): \Gamma \times \Omega \rightarrow [0, +\infty)
\end{equation}

Across multiple time steps $T$, the Learner will suffer multiple individual losses which can be referred to collectively as the cumulative loss up until time $T$ defined below. The Learner's performance is effectively measured by this cumulative loss, so their natural objective is to try to suffer as low a cumulative loss ass possible.

\begin{equation}
    \text{Loss}_L(T) = \underset{t=1}{\overset{T}{\sum}} \lambda(\gamma_t, \omega_t)
\end{equation}

\paragraph{Games}\label{paragraph:games}
With the necessary notation established, we can now delve into the related concept of a \textbf{\textit{Game}} which introduces a strategic perspective to On-line Prediction.

Formally, a Game $G$ is denoted with the triple $\langle \Gamma, \Omega, \lambda \rangle$ which refers to a specific prediction space, outcome space and loss function. Informally, it makes sense to refer to this triple as a game because it encapsulates the interactive, yet adversarial, nature of the problem due to conflicting goals of the Learner and Nature which closely resembles the \textbf{\textit{Repeated Game}} framework discussed in Game Theory~\cite{mertens:1990}.

For our purposes, the Learner must perform sequential decision-making and must output a prediction $\gamma$ taken from $\Gamma$ without knowing the true outcome $\omega$ from $\Omega$ in advance which highlights the imbalance in the game and the potential for adversarial games to be played. Given that the Learner's goal is to minimise their cumulative loss, it is natural to assume that Nature's goal is to select outcomes $\omega$s that try to inflict as much cumulative loss on the Learner over time as possible which is similar to how two players in an adversarial game, such as chess, must develop strategies to optimise their performance based on the information available to them and the the actions they have seen from the other player(s).

\textbf{<TODO: INCLUDE EXAMPLES OF GAMES>}

\subsubsection{Prediction with Expert Advice}\label{subsubsection:prediction_with_expert_advice}
With the foundations of On-line Prediction established, we can now focus on a more nuanced approach to prediction: Prediction with Expert Advice. The approach builds on the existing framework presented in Protocol~\ref{on-line_prediction_framework} by introducing a critical element in the form of the aggregation of multiple Experts' opinions with the central idea being that, by harnessing the insights and predictions from a pool of Experts, the Learner can make more informed and accurate predictions, particularly in complex scenarios with high levels of uncertainty and/or incomplete information.

\begin{protocol}[ht]
    \caption{Prediction with Expert Advice Framework}\label{protocol:prediction_with_expert_advice}
    \begin{algorithmic}[1]
        \State{FOR $t = 1, 2, \ldots$}
        \State{\hspace{\algorithmicindent} experts $\mathcal{E}_1, \ldots, \mathcal{E}_N$ output predictions}$\gamma^1_t, \ldots, \gamma^N_t \in \Gamma$
        \State{\hspace{\algorithmicindent} learner $L$ outputs $\gamma_t \in \Gamma$}
        \State{\hspace{\algorithmicindent} nature outputs $\omega_t \in \Omega$}
        \State{\hspace{\algorithmicindent} experts $\mathcal{E}_1, \ldots, \mathcal{E}_N$ suffer losses $\lambda(\gamma^1_t, \omega_t), \ldots, \lambda(\gamma^N_t, \omega_t)$}
        \State{\hspace{\algorithmicindent} learner $L$ suffers loss $\lambda(\gamma_t, \omega_t)$}
        \State{END FOR}
    \end{algorithmic}
\end{protocol}

As shown above, a Learner in the Prediction with Expert Advice Framework has access to a pool of Experts, denoted $\mathcal{E}_1, \ldots, \mathcal{E}_N$, to help inform their predictions. At each time step $t$, each of the Experts provides the Learner with a prediction $\gamma^1_t, \ldots, \gamma^N_t$ which the Learner must combine to form a single prediction $\gamma_t$.

The theoretical basis for this framework lies in the theory that no single Expert, no matter how knowledgeable, can consistently outperform a well-constructed aggregate of advice from multiple Experts. While the Learner's objective in this scenario is still to minimise the amount of cumulative loss they suffer over time, the manner in which they achieve this is influenced by both the accuracy of each individual Expert, as well as how well each Expert's predictions are incorporated into their aggregated prediction.

In non-adversarial scenarios, it is safe to assume that the Experts' goal is to minimise the individual cumulative losses over time much like the Learner, however, we cannot safely assume that every Expert in the pool will behave in such a manner. Because of this, we must develop a framework to essentially treat each Expert in the pool as a black box, meaning that the Learner has no knowledge of the internal prediction mechanisms of each Expert nor what each individual Expert's goal is, to guarantee that the cumulative loss suffered by the Learner is almost as good as the cumulative loss suffered by the best Expert.

\begin{equation}
    \forall n, \forall T: \text{Loss}_L(T) \lesssim \text{Loss}_{E_n}(T)
\end{equation}

\paragraph{Mixability}\label{paragaph:mixability}
Before we continue to delve into potential solutions for the problem of Prediction with Expert Advice, it is vital that we introduce another vital concept known as \textbf{\textit{mixability}} which refers to a property of the loss function that allows for efficient aggregation of probabilistic predictions.

More precisely, a loss function $\lambda(\gamma_t, \omega_t): \Gamma \times \Omega \rightarrow [0, +\infty)$ is called mixable if there exists a constant $\eta > 0$ such that there exists an aggregated (``mixed'') prediction $\gamma \in \Gamma$ that satisfies the following inequality for any set of probabilistic predictions $\{\gamma^i\}^N_{i=1}$ over the outcomes in $\Omega$:

\begin{equation}
    \lambda(\gamma_t, \omega_t) \leq - \frac{1}{\eta}\underset{n=1}{\overset{N}{\sum}} p^n e^{-\eta\lambda(\gamma^n_t, \omega_t)}
\end{equation}

The significance of this is that it allows the Learner to effectively combine the predictions from multiple sources, ensuring that the aggregated predictions suffer a loss that is close, if not equal, to the best possible weighted combination of individual predictions. Ultimately, this means that mixable prediction strategies can compete with, or outperform, any fixed prediction strategy with hindsight.

With that mentioned, we can now discuss the mathematical methods that can be used to combine the predictions from multiple Experts, namely the Halving, Weighted Majority, and ultimately Aggregating Algorithms. Each of these algorithms aims to leverage the strength of each Expert's predictions to make the Learner's own more accurate while also minimising the impact of inaccurate predictions by hedging so that, in the event that adversarial experts are present, the Learner wouldn't suffer great cumulative loss due to poor reliability.

\paragraph{Halving Algorithm}\label{paragraph:halving_algorithm}
Given that we are now looking at problems involving a pool of Experts, consider the \textbf{\textit{simple prediction game}}, i.e. $\langle \Gamma = \Omega = \{0, 1\}, \lambda_\text{SQ}(\gamma_t, \omega_t) = {(\gamma_t - \omega_t)}^2 \rangle$, where we know in advance that there is a perfect Expert that always gives the correct prediction. Despite the fact that the perfect Expert will not make mistakes, we as the Learner still might since we do not know which of the $N$ Experts is perfect, thus a strategy must be devised and followed to locate them while suffering as little cumulative loss as possible.

Since we know in advance that there is a perfect Expert, we do not need to pay attention to any Experts that we have seen to make mistakes. In order to do this, we need to create and maintain two lists: a \textbf{\textit{whitelist}} and a \textbf{\textit{blocklist}}.

Initially, all $N$ Experts begin in the whitelist and make predictions on every time step $t$ as before, however, as soon as an Expert makes a mistake, i.e., $\gamma^n_t \neq \omega_t$, it is moved to the blocklist with no way of returning to the whitelist. This means that on each time step $t$, we only consider the predictions of experts that are currently a member of the whitelist with the simplest strategy for the Learner being to listen to the majority of Experts' predictions.

The size of the whitelist on after step $t$ is denoted by $W_t$ and initially all Experts are in the whitelist, i.e. $W_0 = N$. By using this method, whenever the Learner makes a mistake, the size of the whitelist shrinks.

Say that there is an error on step $t$. Before this step, the size of the whitelist was $W_{t-1}$ and because we made a mistake, this means that at least half of the experts on the whitelist also made a mistake and are therefore removed, meaning $W_{t} \leq \frac{W_{t-1}}{2}$. More generally, consider that by time $T$ the Learner has made $m$ mistakes, meaning that $W_T \leq \frac{W_0}{2^m}$ with $W_T \geq 1$ because of the presence of at least one perfect Expert.

\begin{equation}
    1 \leq W_T \leq \frac{W_0}{2^m} = \frac{N}{2^m}
\end{equation}

Meaning that a Learner following the Halving Algorithm in the Simple Prediction Game will achieve:
By rearranging this formula, we get that a Learner following the Halving Algorithm in the Simple Prediction game will achieve the following for every sequence of outcomes provided that there is a perfect expert.

\begin{equation}
    \text{Loss}_L(T) \leq \lfloor \log_2 N \rfloor
\end{equation}

\paragraph{Weighted Majority Algorithm}\label{paragraph:weighted_majority_algorithm}
Consider now an extension to the Halving Algorithm known as the Weighted Majority Algorithm which can be played on the \textbf{\textit{discrete binary prediction game}}, i.e. $\langle \Gamma = [0, 1], \Omega = \{0, 1\}, \lambda_\text{SQ}(\gamma_t, \omega_t) = {(\gamma_t - \omega_t)}^2 \langle$ and removes the need for a perfect Expert. To do so, we have to introduce a weight for each expert at each time step, denoted as $w^n_t$, that effectively represents how much the Learner trusts the expert and listens to their advice.

Let each Expert begin with an initial weight $W^n_0 = 1$. Now, instead of being moved from the whitelist to the blocklist whenever an Expert makes a mistake, their weights are updated as follows:

\begin{equation}
    w^n_t = w^n_{t-1} \beta^{\lambda(\gamma^n_t, \omega_t)}
\end{equation}

where $\beta = e^{-\eta} < 1$, with $\eta > 0$ representing a ``learning rate'' which determines how severely the trust in an expert is eroded upon making a mistake. If an expert does not make a mistake, then $\lambda(\gamma^n_t, \omega_t) = 0$ meaning that $e^{-\eta \lambda(\gamma^n_t, \omega_t)} = 1$.

With this notation, it can be seen how the Halving Algorithm is actually a special case of the Weighted Majority Algorithm where $\beta = e^{-\eta} = 0$ as the trust in an Expert gets reduced to zero when they make a mistake.

\begin{algorithm}[ht]
    \caption{Weighted Majority Algorithm}\label{weighted_majority_algorithm}
    \begin{algorithmic}[1]
        \State{initialise weights $w^i_0 = 1, i = 1, 2, \ldots, N$}
        \State{FOR $t = 1, 2, \ldots$}
        \State{\hspace{\algorithmicindent} read the experts' predictions $\gamma^i_t, i=1, 2, \ldots, N$}
        \State{\hspace{\algorithmicindent} calculate the sum of weights $v^0_{t} = \sum_{n : \gamma^n_t = 0} w^n_{t-1}$ \newline\hspace*{\algorithmicindent}\hspace{\algorithmicindent} and $v^1_{t} = \sum_{n : \gamma^n_t = 1} w^n_{t-1}$}
        \State{\hspace{\algorithmicindent} if $v^0_t > v^1_t$, predict $\gamma_t = 0$; otherwise predict $\gamma_t = 1$}
        \State{\hspace{\algorithmicindent} observe the outcome $\omega_t$}
        \State{\hspace{\algorithmicindent} update the experts' weights $w^i_t = w^i_{t-1} \beta^{\lambda(\gamma^i_t, \omega_t)}, i = 1, 2, \ldots, N$}
        \State{END FOR}
    \end{algorithmic}
\end{algorithm}

The upper bound for a Weighted Majority Algorithm is as follows:

\begin{equation}
    \text{Loss}_L(T) \leq \frac{\ln (\frac{1}{\beta})}{\ln (\frac{2}{1 + \beta})} \text{Loss}_{E_n}(T) + \frac{\ln(N)}{\ln(\frac{2}{1+\beta})}
\end{equation}

The proof for this bound is more complicated than the proof for the Halving Algorithm and, thus, will not be discussed in this paper but can be found at~\cite{littlestone:1994}.

In the next subsection, we will explore the Aggregating Algorithm, the final method of merging Experts' predictions and a concrete implementation of the Prediction with Expert Advice that formalises how a Learner's predictions are made as accurate as possible.

\subsubsection{Aggregating Algorithm (AA)}\label{subsubsection:aggregating_algorithm}
Having now established the value of Expert advice when attempting to make predictions, we must now explore how to effectively integrate it into our predictive models. The Aggregating Algorithm is one of the key algorithms that addresses this challenge, providing a way to combine various Expert opinions into a single, actionable prediction that is informed via weighting.

Much like the Weighted Majority Algorithm discussed previously, the Aggregating Algorithm works by maintaining a set of weights for each Expert, representing the Learner's confidence in the accuracy of their predictions. As predictions are made over time, these weights are adjusted based on the loss associated with each expert's predictions with Experts that output predictions closer to the actual outcomes being rewarded with higher weights and Experts that perform poorly being punished with their weights reduced.

\noindent\rule{\textwidth}{0.1pt}

\textbf{Algorithm Description:} Introduction to the Aggregating Algorithm.\newline
\textbf{Functionality:} How the Aggregating Algorithm works in practice.\newline
\noindent\rule{\textwidth}{0.1pt}
\begin{algorithm}[ht]
    \caption{Aggregating Algorithm (AA)}\label{algorithm:aggregationg_algorithm}
    \begin{algorithmic}[1]
        \State{initialise weights $w^i_0 = q_i, i = 1, 2, \ldots, N$}
        \State{FOR $t = 1, 2, \ldots$}
        \State{\hspace{\algorithmicindent} read the experts' predictions $\gamma^i_t, i=1, 2, \ldots, N$}
        \State{\hspace{\algorithmicindent} normalise the experts' weights $p^i_{t-1} = w^i_{t-1} / \sum^N_{j=1} w^j_{t-1}$}
        \State{\hspace{\algorithmicindent} output $\gamma_t \in \Gamma$ that satisfies the inequality for all $\omega \in \Omega$:\newline\hspace*{\algorithmicindent}\hspace{\algorithmicindent} $\lambda(\gamma_t, \omega) \leq - \frac{C}{\eta} \ln \sum^N_{i=1}p^i_{t-1}e^{-\eta\lambda(\gamma^i_t, \omega)}$}
        \State{\hspace{\algorithmicindent} observe the outcome $\omega_t$}
        \State{\hspace{\algorithmicindent} update the experts' weights $w^i_t = w^i_{t-1} e^{-\eta \lambda(\gamma^i_t, \omega_t)}, i = 1, 2, \ldots, N$}
        \State{END FOR}
    \end{algorithmic}
\end{algorithm}

\begin{equation}
    \text{Loss}_T(L) \leq C \cdot \text{Loss}_T(\mathcal{E}_i) + \frac{C}{\eta}\ln\frac{1}{q_i}
\end{equation}

\subsubsection{Aggregating Algorithm for Specialist Experts (AASE)}\label{subsubsection:aggregating_algorithm_for_specialist_experts}
While the Aggregating Algorithm provides a robust framework for incorporating Expert advice, there are certain scenarios that require a more nuanced approach. This is where the Aggregating Algorithm for Specialist Expert comes into play, which will be the approach that this study's experiment is centred.

The use of the term `Specialist' was first introduced by the work of Avrim Blum~\cite{blum:1997} for the Winnow and Weighted-Majority algorithms, and can be thought of as a natural extension to traditional Experts insofar as it enables these `Specialists' to abstain from making a prediction ``when the current Expert does not fall into their `[speciality]'\''. While the criteria for an Expert to abstain from making a prediction is sufficient in our context, it can also be extended to allow for other scenarios like those suggested in~\cite{kalnishkan:2022}, namely if ``a prediction algorithm [sees] that its internal confidence is low and [decides] to skip a turn in order to re-train'' or if a prediction algorithm breaks down, as would be the case if a regression algorithm ``[has] its matrix very close to singular.''

In order to accommodate these Specialist Experts, the Prediction with Expert Advice Framework given in (\ref{on-line_prediction_framework}) has to be modified as follows:
\begin{protocol}[H]
    \caption{Modified Prediction with Expert Advice Framework}\label{protocol:modified_prediction_with_expert_advice}
    \begin{algorithmic}[1]
        \State{FOR $t = 1, 2, \ldots$}
        \State{\hspace{\algorithmicindent} nature chooses a subset of experts $\mathcal{E}_i \in \mathcal{E}$} that are awake
        \State{\hspace{\algorithmicindent} awake experts $\mathcal{E}_1, \ldots, \mathcal{E}_N$ output predictions }$\gamma^1_t, \ldots, \gamma^N_t \in \Gamma$
        \State{\hspace{\algorithmicindent} learner $L$ outputs $\gamma_t \in \Gamma$}
        \State{\hspace{\algorithmicindent} nature outputs $\omega_t \in \Omega$}
        \State{\hspace{\algorithmicindent} awake experts $\mathcal{E}_1, \ldots, \mathcal{E}_N \in \mathcal{E}_i$ suffer losses $\lambda(\gamma^1_t, \omega_t), \ldots, \lambda(\gamma^N_t, \omega_t)$}
        \State{\hspace{\algorithmicindent} learner $L$ and sleeping experts $\mathcal{E}_j \notin \mathcal{E}_i$ suffers loss $\lambda(\gamma_t, \omega_t)$}
        \State{END FOR}
    \end{algorithmic}
\end{protocol}

As referenced above, another colloquial way of referring to `Specialist Experts' is `sleeping Experts'; Freund postulated that ``a Specialist is awake when it makes a prediction and that it is asleep otherwise'', going so far as to refer to the traditional On-line Prediction framework as ``the insomniac framework since it is a special case in which all Specialists are awake all the time.''~\cite{freund:1997} This colloquialism is useful when adapting the bounds of the base Aggregating Algorithm because a natural interpretation of what happens when an Expert is sleeping is that it simply ``joins the crowd''~\cite{kalnishkan:2022}, meaning that it mimics the learner's prediction on the time steps that it is asleep because the learner's prediction is formed based on the weighted majority of Experts' predictions. Given this definition, it can be seen that on some time steps $t$, the learner's prediction and the Expert $\mathcal{E}_i$'s predictions are the same; $\gamma_t = \gamma_t^i$.
Recall that, in the mixable case, the Aggregating Algorithm guarantees that the following inequality is satisfied:

\begin{equation*}
    \overset{T}{\underset{t=1}{\sum}}\lambda(\gamma_t, \omega_t) \leq \overset{T}{\underset{t=1}{\sum}}\lambda(\gamma_t^i, \omega_t) + \frac{1}{\eta} \ln \frac{1}{q_i}
\end{equation*}

Typically, the Aggregating Algorithm's performance is measured in terms of the learner's cumulative loss compared to the best Expert's cumulative loss but given that, on certain time steps $t$, $gamma_t = \gamma_t^i$, it is clear that the corresponding terms in both sums cancel out and what is left are the sums over the time steps where the learner's and the Expert's predictions are different, i.e.\ where Expert $\mathcal{E}_i$ is awake. What follows from this is that, instead of wanting the learner's loss to be nearly as good as the best Expert's loss over a period of time $T$, we judge the Aggregating Algorithm for Specialist Experts' performance based on the learner's loss compared to the best Expert's $\mathcal{E}_i$ loss over the steps in which it was awake. A learner following the algorithm achieves a cumulative loss that satisfies the following inequality:

\begin{equation}
    \overset{T}{\underset{\substack{t=1,2,\ldots,T:\\\mathcal{E}_i\text{ is awake}\\\text{on step }t}}{\sum}}\lambda(\gamma_t, \omega_t) \leq C \cdot \overset{T}{\underset{\substack{t=1,2,\ldots,T:\\\mathcal{E}_i\text{ is awake}\\\text{on step }t}}{\sum}} \lambda(\gamma^i_t, \omega_t) + \frac{C}{\eta}\ln\frac{1}{q_i}
\end{equation}

As is the case for the traditional Aggregating Algorithm, we make no assumptions about the outcome-generating mechanism (including the existence of such a mechanism) and this bound holds for \textit{any} adversarial strategy, meaning that the adversary cannot inflict a large loss on the learner without inflicting a large loss on the Specialists and ensuring that the performance will be good whenever there is a good mixture of Specialists.

\begin{algorithm}[ht]
    \caption{Aggregating Algorithm for Specialist Experts (AASE)}\label{algorithm:aggregating_algorithm_for_specialist_experts}
    \begin{algorithmic}[1]
        \State{initialise weights $w^i_0 = q_i, i = 1, 2, \ldots, N$}
        \State{FOR $t = 1, 2, \ldots$}
        \State{\hspace{\algorithmicindent} read the awake experts' predictions $\gamma^i_t, i=1, 2, \ldots, N$}
        \State{\hspace{\algorithmicindent} normalise the awake experts' weights\newline\hspace*{\algorithmicindent}\hspace{\algorithmicindent}$p^i_{t-1} = w^i_{t-1} / \sum_{j:\mathcal{E}_j\text{ is awake}} w^j_{t-1}$}
        \State{\hspace{\algorithmicindent} output $\gamma_t \in \Gamma$ that satisfies the inequality for all $\omega \in \Omega$:\newline\hspace*{\algorithmicindent}\hspace{\algorithmicindent} $\lambda(\gamma_t, \omega) \leq - \frac{C}{\eta} \ln \sum_{i:\mathcal{E}_i\text{ is awake}}p^i_{t-1}e^{-\eta\lambda(\gamma^i_t, \omega)}$}
        \State{\hspace{\algorithmicindent} observe the outcome $\omega_t$}
        \State{\hspace{\algorithmicindent} update the awake experts' weights $w^i_t = w^i_{t-1} e^{-\eta\lambda(\gamma^i_t, \omega_t)}$}
        \State{\hspace{\algorithmicindent} update the sleeping experts' weights $w^i_t = w^i_{t-1} e^{-\eta\lambda(\gamma_t, \omega_t)/ C(\eta)}$}
        \State{END FOR}
    \end{algorithmic}
\end{algorithm}

\subsection{Conclusion}
Having gone through the established knowledge base on these topics, this Literature Review has now traced how the perception of randomness in both judging and generating random sequences has changed over time, highlighting key studies that have informed the current understanding of these cognitive processes with the conflicting findings highlighting the complexity of the task. Ultimately, the disagreement suggests that while humans may not be inherently good randomisers individually when compared to the statistical definition of random, their performance can vary significantly depending on the conditions in which they are tested with an interesting observation being made in that when sequences generated by multiple individuals are aggregated, the distribution becomes more like what is expected of a truly random process.

In the context of Prediction with Expert Advice, these insights are particularly interesting because the imperfections of human-generated ``random'' sequences, primarily marked by the tendency to favour sequences with a greater number of alternations, or that contain longer runs than what is expected, might actually be more predictable than realised by making use of On-line Prediction algorithms, providing evidence for the fact that humans are, in fact, bad randomisers.

By contextualising the experiment conducted in this study with the frameworks established by various papers cited within this Chapter, this Literature Review sets the stage for the subsequent analysis of human-generated ``random'' sequences with the findings from the experiment assisting in providing a deeper understanding of the human perception of randomness and the potential implications of that on predictive computational models.