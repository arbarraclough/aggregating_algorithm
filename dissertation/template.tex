\documentclass[11pt]{article} % do not change this line
\input{BigDataStyle.txt}      % do not change this line
\usepackage{algorithm,algpseudocode,amsfonts,amssymb,amsthm,amsmath,graphicx,hyperref,latexsym,nicefrac,subcaption,titlesec}

\emergencystretch=5mm
\tolerance=400
\allowdisplaybreaks[4]

\theoremstyle{plain}
\newtheorem{theorem}{Theorem}[section]
\newtheorem{proposition}[theorem]{Proposition}
\newtheorem{corollary}[theorem]{Corollary}
\newtheorem{lemma}[theorem]{Lemma}
\newtheorem{problem}[theorem]{Problem}

\theoremstyle{definition}
\newtheorem*{remark}{Remark}

\title{Aggregating Algorithm}
\author{Candidate 2408208}

\newcommand{\Programme}{Machine Learning}
% Computational Finance students: uncomment the next line
%\twodepartmentstrue

\newcounter{protocol}
\makeatletter
\newenvironment{protocol}[1][htb]{
  \let\c@algorithm\c@protocol{}
  \renewcommand{\ALG@name}{Protocol}
  \begin{algorithm}[#1]
  }{\end{algorithm}
}
\makeatother

\setcounter{secnumdepth}{4}
\titleformat{\paragraph}
{\normalfont\small\bfseries}{\theparagraph}{1em}{}
\titlespacing*{\paragraph}
{0pt}{3.25ex plus 1ex minus .2ex}{1.5ex plus .2ex}

\titleformat{\subparagraph}
{\normalfont\footnotesize\itshape}{\theparagraph}{1em}{}
\titlespacing*{\subparagraph}
{0pt}{3.25ex plus 1ex minus .2ex}{1.5ex plus .2ex}

\begin{document}
\maketitle

\declaration{}

\begin{abstract}
  This dissertation investigates how well humans generate objectively random binary sequences by evaluating human-generated sequences against true randomness using a variety of statistical methods, as well as assessing the predictability of such sequences using the Aggregating Algorithm, inspired by Markov Chains.
  
  The study is designed with a within-subject approach and makes use of a web application to allow subjects\textemdash{}primarily postgraduate students at Royal Holloway, University of London\textemdash{}to input binary sequences sequentially. As the subject enters their sequence, the bits are analysed by various ``Experts'', each of whom make their own predictions on the next bit in the sequence prior to the subject pressing a key. These predictions are then mixed using the Aggregating Algorithm to form a single prediction which is then compared against the observed outcome entered by the subject.
  
  This study measured various dependent variables, including the frequences of 0s and 1s (Heads and Tails), run lengths, and predictions from a variety of experts as well as the algorithm itself. The data gathered was evaluated using chi-square goodness-of-fit tests and by analysing the regret associated with each Expert/Learner to compare the generated sequences to statistically random processes. In summary, the Aggregating Algorithm's performance was evaluated, revealing that human-generated sequences do suffer from biases and are, therefore, predictable to a statistically significant extent better than is possible with random guessing.\newline\newline
  \textbf{Keywords:} On-line Learning, Prediction with Expert Advice, Aggregating Algorithm for Specialist Experts, Perceived Randomness
\end{abstract}

\section*{Acknowledgements}
While the contents of this report are based on my work, none of this would have been possible without the patience and mentorship of my supervisor to whom I am extremely grateful. It was your advice, clear explanations, and expertise that made this project what it is now and something that I am incredibly proud of.
I would also like to express my gratitude to the group of friends who made this academic year possible, namely Cougar Tasker, Einstein Ebereonwu, Hayden Amarr, Mohammadreza Yazdian, Niraj Jain, and Ray Mahbub, without whom I would have struggled to maintain my discipline and motivation.
\newpage

\section{Introduction \textit{(1,000)} \textbf{(728)}}\label{section:introduction}

\subsection{Project Scope and Objectives \textbf{(205)}}
The aim of this project is to implement methods of Prediction with Expert Advice, such as the Aggregating Algorithm, and to evaluate their performance in different scenarios, specifically targeting  real-world applications. 

As an introduction to the concepts explored in the chapters to come, these methods allow for the pooling of different prediction algorithms (known as `experts') with the goal of improving prediction accuracy---allowing the final prediction to be nearly as accurate as the best-performing expert.

This project will encompass several key areas, including:
\begin{itemize}
    \item \textbf{Explaining the Theory of Prediction with Expert Advice.} To effectively experiment with different methods of Prediction with Expert Advice, the underlying theory must first be understood by conducting a review of the relevant literature.
    \item \textbf{Implementing the Aggregating Algorithm.} This project will primarily investigate the Aggregating Algorithm introduced by Vovk (see~\cite{vovk:1990},\ \cite{vovk:1998}).
    \item \textbf{Handling Specialist Experts.} Introduced by Freund~\cite{freund:1997}, \textit{Specialist Experts} may refrain from making predictions at certain points, meaning that the Aggregating Algorithm has to be modified slightly~\cite{kalnishkan:2015}.
    \item \textbf{Applying Prediction with Expert Advice to Real-World Data.} The methods described in this report will be applied to real-world datasets in order to evaluate their practicality outside of theoretical models, including an investigation into the perception of randomness by utilising specialist experts.
\end{itemize}

\subsection{Motivation and Interest in the Subject Area \textbf{(261)}}
The motivation for selecting a project in this subject area is rooted in both my personal and professional interests, as well as the discussions I had with my now-supervisor, Dr.\ Yuri Kalnishkan, before finalising my selection.

During this academic year, the module that most piqued my interest was CS5200 \textendash\ On-line Machine Learning because I was interested in the techniques that allowed machine learning models to gradually improve over time as more data became available to them without the need to retrain the model on the entire newly-updated dataset; something that had not been covered previously by other modules. Due to the module's small size and frequent absentees, I was able to gain a deeper understanding of the module, in large part due to Dr.\ Kalnishkan's willingness to explain portions of the syllabus in extreme detail. Alongside the lectures, I felt like I was strongly suited to the contents of the module because it has strong ties to the field of statistics \textendash\ another area that I thoroughly enjoyed throughout my education. 

Regarding my professional aspirations, I am set to begin my career later this year and I am of the firm belief that the work that I have done in this subject area is highly relevant, not only to the job I am to start in September, but also for my career plan due to its relevance across a variety of industries \textendash\ including finance, energy, and insurance.

Ultimately, the combination of all of these factors led me to pursue a project investigating on-line prediction, and prediction with expert advice.

\subsection{Structure of the Dissertation \textbf{(262)}}
The dissertation is split into distinct chapters, each dedicated to exploring a specific aspect of the work. The following outline guides the reader through the report by providing a brief overview of the contents of each chapter.

Chapters 2 through 5 contain a literature review organised to explain the concepts that the practical portion of the dissertation aims to explore. \hyperref[section:On-line_Prediction]{Chapter 2} defines the problem of On-line Prediction, outlining the scenarios that it applies to, and the protocols that such problems follow. Additionally, it explores how on-line learning differs from traditional batch learning and defines concepts that will be critical to understanding the following sections. \hyperref[section:Prediction_with_Expert_Advice]{Chapter 3} introduces the problem of Prediction with Expert Advice, explaining its significance and applications in the real world, as well as exploring some algorithms that are used to solve such problems \textendash\ including their theoretical bounds. \hyperref[section:Aggregating_Algorithm]{Chapter 4} introduces the Aggregating Algorithm that this report is centred around, exploring how it differs from other methods of Prediction with Expert Advice. \hyperref[section:Specialist_Experts]{Chapter 5} focuses on Specialist Experts, defining what they are and how the base Aggregating Algorithm must be modified to accommodate them.

\hyperref[section:Practical]{Chapter 6} contains the practical portion of the dissertation, explaining how the research problem was handled based on the concepts explored in the literature review, the findings from conducting the requirements analysis and design processes, and the results found when comparing an individual's idea of ``random'' to that of a random number generator.

Finally, \hyperref[section:Conclusion]{Chapter 7} contains a conclusion which discusses the findings of the investigation as well as a self-evaluation of the project.
\newpage

\section{Literature Review}\label{section:Literature_Review}
\subsection{Introduction}
\textbf{Purpose:} Overview of the goals of the literature review.\newline
\textbf{Scope:} Outline of the topics covered and their relevance to the dissertation.\newline
\noindent\rule{\textwidth}{0.1pt}

\newpage

\subsection{Perceived Randomness}
\subsubsection{Introduction to Perceived Randomness}
\textbf{Definition:} Explanation of what perceived randomness is.\newline
\textbf{Importance:} Discussion on why perceived randomness is significant in various fields.\newline
\noindent\rule{\textwidth}{0.1pt}

\subsubsection{Randomness in Binary Sequences}
\textbf{Human vs. Algorithmic Generation}\newline
\textbf{Human Perception:} How humans perceive randomness.\newline
\textbf{Algorithmic Methods:} Comparison of human and algorithmic sequence generation.\newline
\noindent\rule{\textwidth}{0.1pt}

\noindent\textbf{Methods for Generating Sequences}\newline
\textbf{Techniques:} Different methods for generating binary sequences.\newline
\textbf{Comparative Analysis:} Evaluation of these methods in terms of perceived randomness.\newline
\noindent\rule{\textwidth}{0.1pt}

\newpage

\subsection{Prediction with Expert Advice}
\subsubsection{Introduction to On-line Prediction}

% \textbf{Concept:} Overview of on-line prediction in the context of perceived randomness.\newline
% \textbf{Applications:} Various applications of on-line prediction techniques.\newline
% \noindent\rule{\textwidth}{0.1pt}

\noindent Within the areas of Machine Learning and Statistics, there lies the problem of accurately ``predicting future events based on past observations''~\cite{cesa-bianchi:1997}  known as \textit{on-line prediction}. This problem refers to methods where a model makes predictions sequentially and updates its parameters in real-time as new data points become available. There is a particular class of algorithm that is designed to tackle this, with one of the most notable being the ``Strong'' Aggregating Algorithm proposed by Volodymyr Vovk~\cite{vovk:1990} which forms the basis of this study. The adjective ``Strong'' is emphasised with inverted commas to help distinguish the algorithm from the ``Weak'' Aggregating Algorithm proposed by Yuri Kalnishkan and Michael Vyugin~\cite{kalnishkan/vyugin:2008} that will be touched upon but not explored in detail in this dissertation.

Given that the foundations of this dissertation lie firmly in this subject area, this section aims to lay a comprehensive foundation, exploring the key concepts and frameworks that will set the stage for the discussions in Chapter \textbf{TODO}.

\paragraph*{On-line Prediction, Batch Learning and Timeseries Analysis}
Herein the first distinction between on-line prediction and the traditional batch learning framework. With batch learning, a whole training set of labelled examples $(x_i, y_i)$ is given to the learner at once in order to train a model. In contrast, on-line learning involves gradually feeding the learner information over time, requiring the model to continuously adapt to the new data it is given while requiring the learner to take actions on the basis of the information it already possesses instead of waiting for a complete picture.~\cite{kalnishkan:2015} This forced adaptability ensures that the predictions outputted by the algorithm remain accurate based on the information that the model deems as relevant as it gains additional knowledge, making these models particularly valuable in applications that require immediate responses and fluidity such as financial market analysis and weather forecasting.

Another distinction that needs to be made is between on-line prediction and timeseries analysis as, while these are both ways of handling sequential data in machine learning and statistics, they are unique. On-line learning is based on processing data points sequentially and updating predictive models in real-time whereas timeseries analysis is based on modelling and forecasting data that is collected over successive time intervals. The prior approach does not impose any strict assumptions about the underlying data-generating process, even going so far as to not assume the existence of such a process~\cite{vovk:2001}, while the latter assumes a structured approach where observations are dependent on previous observations. These are typically modelled using stochastic processes such as \textit{autoregressive integrated moving average (ARIMA)} or \textit{state-space} models~\cite{box:2015}. The majority of the literature on On-line Prediction takes a similar stance that no assumptions can be made about the sequence of outcomes that are observed. Because of this, the analyses are done over the worst-case and may be better in reality~\cite{cesa-bianchi:1997}.

\paragraph*{Notation}
In on-line prediction, we consider a scenario where the elements of a sequence, known as \textit{\textbf{outcomes}}, $\omega_t$ occur at discrete times $\omega_1, \omega_2, \ldots$ which we assume to be drawn from a known \textit{\textbf{outcome space}} $\Omega$. In this problem, a learner is tasked with making \textit{\textbf{predictions}} $\gamma_t$ about these \textit{outcomes} one at a time before they occur. Similarly, we assume that the learner's predictions are drawn from a known \textit{\textbf{prediction space}} $\Gamma$ which may or may not be the same as the \textit{outcome space} $\Omega$. 

Once the learner has made their \textit{prediction}, the true \textit{outcome} is then revealed and the quality of the learner's prediction is assessed by a \textit{\textbf{loss function}} $\lambda(\gamma_t, \omega_t)$. This function measures the discrepancy between the \textit{prediction} and \textit{outcome} or, more generally, quantifies the effect of when the \textit{prediction} $\gamma_t$ is confronted with the \textit{outcome} $\omega_t$~\cite{adamskiy:2019} by mapping the input space $\Gamma \times \Omega$ to a subset of the real-number line $\mathbb{R}$, typically $[0, +\infty)$~\cite{kalnishkan:2009}.

Across several time steps $T$, the learner will suffer multiple losses which can be referred to as their cumulative loss up to time $T$. Their performance is measured by this cumulative loss, so their natural objective is to suffer as low a cumulative loss as they can.

\begin{protocol}[H]
    \caption{On-line Prediction Framework}\label{on-line_prediction_framework}
    \begin{algorithmic}[1]
        \State{FOR $t = 1, 2, \ldots$}
        \State{\hspace{\algorithmicindent} learner $L$ outputs $\gamma_t \in \Gamma$}
        \State{\hspace{\algorithmicindent} nature outputs $\omega_t \in \Omega$}
        \State{\hspace{\algorithmicindent} learner $L$ suffers loss $\lambda(\gamma_t, \omega_t)$}
        \State{END FOR}
    \end{algorithmic}
\end{protocol}

\paragraph{Games and Mixability}
The combination of a \textit{prediction space}, \textit{outcome space}, and \textit{loss function} can be referred to with a triple $<\Gamma, \Omega, \lambda>$, known as a \textit{\textbf{Game}} $G$. \textit{TODO: Explain mixability and touch on (Kalnishkan \& Vyugin, 2008)}~\cite{kalnishkan/vyugin:2008} 

\subsubsection{Prediction with Expert Advice}
\textbf{Framework:} Description of the prediction with expert advice framework.\newline
\textbf{Mechanisms:} Detailed explanation of how this framework operates.\newline
\noindent\rule{\textwidth}{0.1pt}
\begin{protocol}[H]
    \caption{Prediction with Expert Advice Framework}\label{alg:cap}
    \begin{algorithmic}[1]
        \State{FOR $t = 1, 2, \ldots$}
        \State{\hspace{\algorithmicindent} experts $E_1, \ldots, E_N$ output predictions}$\gamma^1_t, \ldots, \gamma^N_t \in \Gamma$
        \State{\hspace{\algorithmicindent} learner $L$ outputs $\gamma_t \in \Gamma$}
        \State{\hspace{\algorithmicindent} nature outputs $\omega_t \in \Omega$}
        \State{\hspace{\algorithmicindent} experts $E_1, \ldots, E_N$ suffer losses $\lambda(\gamma^1_t, \omega_t), \ldots, \lambda(\gamma^N_t, \omega_t)$}
        \State{\hspace{\algorithmicindent} learner $L$ suffers loss $\lambda(\gamma_t, \omega_t)$}
        \State{END FOR}
    \end{algorithmic}
\end{protocol}


\subsubsection{Aggregating Algorithm (AA)}
\textbf{Algorithm Description:} Introduction to the Aggregating Algorithm.\newline
\textbf{Functionality:} How the Aggregating Algorithm works in practice.\newline
\noindent\rule{\textwidth}{0.1pt}
\begin{algorithm}[H]
    \caption{Aggregating Algorithm (AA)}\label{alg:cap}
    \begin{algorithmic}[1]
        \State{initialise weights $w^i_0 = q_i, i = 1, 2, \ldots, N$}
        \State{FOR $t = 1, 2, \ldots$}
        \State{\hspace{\algorithmicindent} read the experts' predictions $\gamma^i_t, i=1, 2, \ldots, N$}
        \State{\hspace{\algorithmicindent} normalise the experts' weights $p^i_{t-1} = w^i_{t-1} / \sum^N_{j=1} w^j_{t-1}$}
        \State{\hspace{\algorithmicindent} output $\gamma_t \in \Gamma$ that satisfies the inequality for all $\omega \in \Omega$:\newline\hspace*{\algorithmicindent}\hspace{\algorithmicindent} $\lambda(\gamma_t, \omega) \leq - \frac{C}{\eta} \ln \sum^N_{i=1}p^i_{t-1}e^{-\eta\lambda(\gamma^i_t, \omega)}$}
        \State{\hspace{\algorithmicindent} observe the outcome $\omega_t$}
        \State{\hspace{\algorithmicindent} update the experts' weights $w^i_t = w^i_{t-1} e^{-\eta \lambda(\gamma^i_t, \omega_t)}, i = 1, 2, \ldots, N$}
        \State{END FOR}
      \end{algorithmic}
\end{algorithm}

\begin{equation}
    \text{Loss}_T(L) \leq C \cdot \text{Loss}_T(\mathcal{E}_i) + \frac{C}{\eta}\ln\frac{1}{q_i} 
\end{equation}

\paragraph{Weak Aggregating Algorithm (WAA)}

\paragraph{Fixed Share Algorithm}

\subsubsection{Aggregating Algorithm for Specialist Experts (AASE)}
\textbf{Specialisation:} Differences between general and specialist experts.\newline
\textbf{Algorithm Adaptation:} How the Aggregating Algorithm is adapted for specialist experts.\newline
\noindent\rule{\textwidth}{0.1pt}
\begin{algorithm}[H]
    \caption{Aggregating Algorithm for Specialist Experts (AASE)}\label{alg:cap}
    \begin{algorithmic}[1]
        \State{initialise weights $w^i_0 = q_i, i = 1, 2, \ldots, N$}
        \State{FOR $t = 1, 2, \ldots$}
        \State{\hspace{\algorithmicindent} read the awake experts' predictions $\gamma^i_t, i=1, 2, \ldots, N$}
        \State{\hspace{\algorithmicindent} normalise the awake experts' weights\newline\hspace*{\algorithmicindent}\hspace{\algorithmicindent}$p^i_{t-1} = w^i_{t-1} / \sum_{j:\mathcal{E}_j\text{ is awake}} w^j_{t-1}$}
        \State{\hspace{\algorithmicindent} output $\gamma_t \in \Gamma$ that satisfies the inequality for all $\omega \in \Omega$:\newline\hspace*{\algorithmicindent}\hspace{\algorithmicindent} $\lambda(\gamma_t, \omega) \leq - \frac{C}{\eta} \ln \sum_{i:E_i\text{ is awake}}p^i_{t-1}e^{-\eta\lambda(\gamma^i_t, \omega)}$}
        \State{\hspace{\algorithmicindent} observe the outcome $\omega_t$}
        \State{\hspace{\algorithmicindent} update the awake experts' weights $w^i_t = w^i_{t-1} e^{-\eta\lambda(\gamma^i_t, \omega_t)}$}
        \State{\hspace{\algorithmicindent} update the sleeping experts' weights $w^i_t = w^i_{t-1} e^{-\eta\lambda(\gamma_t, \omega_t)/ C(\eta)}$}
        \State{END FOR}
    \end{algorithmic}
\end{algorithm}

The learner following this algorithm achieves loss that satisfies:

\begin{equation}
    \overset{T}{\underset{\substack{t=1,2,\ldots,T:\\\mathcal{E}_i\text{ is awake}\\\text{on step }t}}{\sum}}\lambda(\gamma_t, \omega_t) \leq C \cdot \overset{T}{\underset{\substack{t=1,2,\ldots,T:\\\mathcal{E}_i\text{ is awake}\\\text{on step }t}}{\sum}} \lambda(\gamma^i_t, \omega_t) + \frac{C}{\eta}\ln\frac{1}{q_i} 
\end{equation}

\subsection{Conclusions}
\textbf{Summary:} Recap of key points covered in the literature review.\newline
\textbf{Implications:} Implications of the reviewed literature for the current study.\newline
\newpage

\section{Experiment Design and Methodology}\label{section:Experiment_Design_and_Methodology}
\textbf{Research Design:} Overview of the experimental framework.\newline
\textbf{Methodology:} Detailed description of the methods used for data collection and analysis.\newline
\textbf{Variables:} Identification of key variables and how they are measured.\newline
\textbf{Procedures:} Step-by-step outline of the experimental process.\newline
\newpage

\section{Analysis of Perceived Randomness}\label{section:analysis_of_perceived_randomness}
As mentioned in Subsection~\ref{subsection:data_analysis}, the analysis of this study's findings is performed using two methods: chi-square goodness-of-fit, and by comparing the differences in the cumulative loss of the Learner $L$ and each Expert $\mathcal{E}_i \in \{\mathcal{E}_1,\ldots, \mathcal{E}_N\}$. With these methods, this Chapter delves into the results to provide evidence for the hypothesised question ``Are humans good randomisers?'' 

\subsection{Chi-Square Goodness-of-Fit}\label{subsection:chi-square_goodness-of-fit}
We begin with the Chi-Squared ($\chi^2$) Goodness-of-Fit, a statistical method for determining if observed results are similar to what is expected from the null hypothesis\textemdash{}that humans are good randomisers.

\subsubsection{Distribution of the Number of Heads}
Figure~\ref{distribution_of_the_number_of_heads} shows that the distribution produced by the subjects is qualitatively similar to the theoretical distribution in that it is bell-shaped, however the results of $\chi^2(10, N=175) = 155.14, p=3.26\mathrm{e}{-28} < 0.0001$ strongly indicate that the binomial distribution is not a good fit, with large discrepancies found at either extreme, and for sequences with $7+$ heads. As noted in~\cite{nickerson:2009}, ``in a randomly produced set of toss sequences, about 25\% of the sequences should have five heads and about 65\% should have four to six heads.'' In this study's sequences, 34\% had exactly five, and 73\% had four to six heads.

\begin{figure}[h]
    \centering
    \includegraphics[width=0.9\textwidth]{images/combined_number_of_heads.jpg}
    \caption{Percentage of 10-Item Sequences with $n$ Heads, Compared to the Theoretical Distribution, $X \sim B(10, 0.5)$}
    \label{distribution_of_the_number_of_heads}
\end{figure}

Overall, these findings are more than significant enough to reject the null hypothesis and support the conclusion noted by Nickerson and Butler in that ``when trying to emulate a random generator of binary sequences, people are likely to produce a larger percentage of sequences that contain the two elements (e.g., heads and tails) in approximately equal proportions than is a random process.''~\cite{nickerson:2009}

\subsubsection{Distribution of the Number of Runs}
Unlike with the Distribution of the Number of Heads, the distribution produced by the subjects shown in Figure~\ref{distribution_of_the_number_of_runs} is both qualitatively and quantitatively different to what was expected, $\chi^2(9, N=175) = 908.46, p=9.29\mathrm{e}{-190} < 0.0001$. The produced distribution does not have a discernable bell shape, and instead, seems to steadily rise to $r = 8$ before significantly dropping off. 


\begin{figure}[h]
    \centering
    \includegraphics[width=0.9\textwidth]{images/combined_number_of_runs.jpg}
    \caption{Percentage of 10-Item Sequences with $r$ Runs, Compared to the Theoretical Distribution}
    \label{distribution_of_the_number_of_runs}
\end{figure}

Despite looking different to that produced by Nickerson and Butler, this data still supports their conclusion that ``people are likely to fail when trying to emulate a random process \textendash{} in this case by being less likely than a random process to produce sequences with an intermediate number of runs and more likely to produce sequences with close to the minimum or maximum number possible''~\cite{nickerson:2009}. The discrepancies that follow the peak at $r = 8$ also support Bar-Hillel and Wagenaar's conclusion that ``humans produce series with higher than expected alternation rates''~\cite{bar-hillel:1991}.

\subsubsection{Distribution of Run Lengths}
As previously defined, a \textit{run} is a consecutive sequence of the same outcome (either all heads or all tails), meaning that a run length of 1 occurs when only a single head or tail appears before the outcome changes. While the distribution produced by subjects appears qualitatively similar to the theoretical distribution in that the frequency of run lengths decreases exponentially, a significant percentage of runs more than expected had a length of 1, while all other run lengths were underrepresented in the data, $\chi^2(9, N=1,160) = 33.25, p=0.0001$.

\begin{figure}[h]
    \centering
    \includegraphics[width=0.9\textwidth]{images/combined_length_of_runs.jpg}
    \caption{Percentage of Runs with Length $m$, Compared to the Theoretical Distribution}
\end{figure}

Once again, this supports the conclusion drawn by~\cite{bar-hillel:1991} in that human-generated sequences favour alternation over continuation with 90\% of runs across all sequences being up to length $2$, and 97\% being up to length $4$.


\subsection{Regret Analysis}
Given the evidence shown in Subsection~\ref{subsection:chi-square_goodness-of-fit} supporting the conclusion that humans are not good randomisers, we now delve into an evaluation of the Aggregating Algorithm's performance at being able to predict the next bit of human-generated binary sequence.

As previously discussed, a Learner makes use of the predictions from several Experts to form their own about the next bit in a sequence entered by subjects after the fact (acting as Nature in this context). Continuing to assume the null hypothesis that humans are good randomisers, the Aggregating Algorithm should perform no better than randomly guessing the next bit of a sequence, i.e., having a 50\% success rate, however, the plot below shows that this is not the case for any of the subjects tested.

\begin{figure}[h]
    \centering
    \includegraphics[width=0.9\textwidth]{images/prediction_results.jpg}
    \caption{Correct vs. Incorrect Predictions for Each Subject}
    \label{prediction_results}
\end{figure}

The performance of the Aggregating Algorithm exceeded that of random guessing across all subjects, although the improvement was only marginal for Subjects 3 and 4. On average, the Aggregating Algorithm achieved an accuracy of 69.19\%, reinforcing the notion that human-generated sequences exhibit predictable patterns, even when only considering prefixes up to length $4$.

An examination of the difference in cumulative losses between individual Experts and the Learner across all timestamps reveals patterns in the data. When the difference between an Expert's and the Learner's cumulative loss is positive (i.e., the line is above the x-axis), it indicates that the Expert's predictions are less accurate than those of the aggregated model. This implies that the Learner is more effectively aggregating the given predictions, and suggests that the subject is less predictable when inputting a certain prefix, as the relevant Expert's predictions are less accurate.

Conversely, when the difference is negative (i.e., the line is below the x-axis), the individual Expert's predictions outperform the Learner's aggregate. This scenario implies that the Learner's aggregation is less effective and that the subject's behaviour is more predictable for a given prefix as the relevant Expert's predictions are more accurate.

Reviewing the difference plots for all participants, shown below, it is noteworthy that only the Experts designed to detect a Markov Chain of Length 1\textemdash{}a first-order Markov Chain\textemdash{}displayed a difference greater than 2. This observation supports the existing literature: humans tend to avoid immediate repetition and exhibit subconscious longer-term patterns and dependencies that a first-order Markov Chain cannot capture. This is due to inherent memory constraints. For example, if a participant alternates between bits every two or three timestamps, a first-order Markov Chain will be unable to detect the pattern and cause inaccurate predictions,

Furthermore, the results indicate that as a participant's sequences display more qualitative similarities to true randomness, the Aggregating Algorithm's predictive accuracy decreases. This is reflected in a greater proportion of lines closer to the x-axis, as well as alternating signs, signifying an increase in randomness due to the appearance of more inconsistent patterns.

\begin{figure}[hb]
    \centering
    \includegraphics[width=0.625\textwidth]{images/AH_differences.jpg}
    \caption{Cumulative Loss Differences: Experts vs. Learner for Subject 1}
\end{figure}
\begin{figure}[ht]
    \centering
    \includegraphics[width=0.625\textwidth]{images/EE_differences.jpg}
    \caption{Cumulative Loss Differences: Experts vs. Learner for Subject 2}
\end{figure}
\begin{figure}[h!]
    \centering
    \includegraphics[width=0.625\textwidth]{images/HA_differences.jpg}
    \caption{Cumulative Loss Differences: Experts vs. Learner for Subject 3}
\end{figure}
\begin{figure}[ht]
    \centering
    \includegraphics[width=0.625\textwidth]{images/ME_differences.jpg}
    \caption{Cumulative Loss Differences: Experts vs. Learner for Subject 4}
\end{figure}
\begin{figure}[h!]
    \centering
    \includegraphics[width=0.625\textwidth]{images/NJ_differences.jpg}
    \caption{Cumulative Loss Differences: Experts vs. Learner for Subject 5}
\end{figure}
\begin{figure}[ht]
    \centering
    \includegraphics[width=0.625\textwidth]{images/RM_differences.jpg}
    \caption{Cumulative Loss Differences: Experts vs. Learner for Subject 6}
\end{figure}
\begin{figure}[h!]
    \centering
    \includegraphics[width=0.625\textwidth]{images/RY_differences.jpg}
    \caption{Cumulative Loss Differences: Experts vs. Learner for Subject 7}
\end{figure}
\newpage

\section{Conclusions \textit{(1,500)}}\label{section:conclusions}
\textbf{Summary of Findings:} Recap of the main findings of the study.\newline
\textbf{Contributions:} Discussion on the contributions of the study to the field.\newline
\textbf{Limitations:} Identification of any limitations encountered during the research.\newline
\textbf{Future Work:} Suggestions for future research based on the findings and limitations of this study.\newline
\newpage

\bibliographystyle{ieeetr}
\bibliography{bibliography}
\end{document}
